\documentclass[twocolumn,twoside,11pt,a4paper]{article}
\usepackage[utf8]{inputenc}     % 8 bits
\usepackage[pdftex]{graphicx}   % images .png or .pdf w/ pdflatex OR .eps w/ latex
\usepackage[T1]{fontenc}        % PS fonts
\usepackage{lmodern}            % fonts, sudo apt-get install lmodern
\usepackage{parskip}            % no indentation on paragraphs
\usepackage{tabularx}           % more on tables
\usepackage{longtable}          % more pages
\usepackage{url}                % URLs

\usepackage[sc]{mathpazo}       % Use the Palatino font
\linespread{1.05}               % Line spacing - Palatino needs more space between lines
\usepackage{microtype}          % Slightly tweak font spacing for aesthetics
\usepackage[hang, small, labelfont=bf,up,textfont=it,up]{caption} % Custom captions under/above floats in tables or figures
\usepackage{booktabs}           % Horizontal rules in tables
\usepackage{float}              % Required for tables and figures in the multi-column environment - they need to be placed in specific locations with the [H] (e.g. \begin{table}[H])
\usepackage{paralist}           % Used for the compactitem environment which makes bullet points with less space between them

% geometry package
\usepackage[outer=20mm,inner=20mm,vmargin=15mm,includehead,includefoot,headheight=15pt]{geometry}
%% space between columns
\columnsep 10mm

\usepackage{abstract}           % Allows abstract customization
\renewcommand{\abstractnamefont}{\normalfont\bfseries} % Set the "Abstract" text to bold
\renewcommand{\abstracttextfont}{\normalfont\small\itshape} % Set the abstract itself to small italic text

% \usepackage{titlesec}           % Allows customization of titles
% \renewcommand\thesection{\Roman{section}} % Roman numerals for the sections
% \renewcommand\thesubsection{\Roman{subsection}} % Roman numerals for subsections
% \titleformat{\section}[block]{\large\scshape\centering}{\thesection.}{1em}{} % Change the look of the section titles
% \titleformat{\subsection}[block]{\large}{\thesubsection.}{1em}{} % Change the look of the section titles

\usepackage[pdftex]{hyperref}
\hypersetup{%
    a4paper = true,              % use A4 paper
    bookmarks = true,            % make bookmarks
    colorlinks = true,           % false: boxed links; true: colored links
    pdffitwindow = false,        % page fit to window when opened
    pdfpagemode = UseNone,       % do not show bookmarks
    pdfpagelayout = SinglePage,  % displays a single page
    pdfpagetransition = Replace, % page transition
    linkcolor=blue,              % hyperlink colors
    urlcolor=blue,
    citecolor=blue,
    anchorcolor=green
}

\usepackage{indentfirst}         % indent also 1st paragraph

\usepackage{fancyhdr}            % Headers and footers
\pagestyle{fancy}                % pages have headers and footers
\fancyhead{}                     % Blank out the default header
\fancyfoot{}                     % Blank out the default footer
\fancyhead[LO,RE]{\textit{Spotify on the rocks} - cocktails de música} % Custom header text
\fancyhead[RO,LE]{\thepage}      % Custom header text
\fancyfoot[RO,LE]{Grupo 5, \today} % Custom footer text
\renewcommand{\headrulewidth}{0.4pt}
\renewcommand{\footrulewidth}{0.4pt}

%\hyphenation{}                  % explicit hyphenation

%----------------------------------------------------------------------------------------
%	macro definitions
%----------------------------------------------------------------------------------------

% entities
\newcommand{\class}[1]{{\normalfont\slshape #1\/}}
\newcommand{\svg}{\class{SVG}}
\newcommand{\scada}{\class{SCADA}}
\newcommand{\scadadms}{\class{SCADA/DMS}}

%----------------------------------------------------------------------------------------
%	TITLE SECTION
%----------------------------------------------------------------------------------------

% article title
\title{\vspace{-15mm}\fontsize{24pt}{10pt}\selectfont\textbf{Spotify on the rocks}}

% authors
\author{Ana Santos\\
\small \texttt{ansantos3@gmail.com}\\
\and
Rui Pedro Lima\\
\small \texttt{ruipedro.lima@gmail.com}
\vspace{-5mm}
}

\date{\today}

%----------------------------------------------------------------------------------------

\begin{document}

\maketitle
\thispagestyle{plain}            % no headers in the first page

%----------------------------------------------------------------------------------------
%	ABSTRACT
%----------------------------------------------------------------------------------------

\begin{abstract}


\textit{Spotify on the Rocks} é uma aplicação desenvolvida no âmbito da disciplina de Descrição,
Armazenamento e Processamento de Informação (DAPI), disciplina do 1º semestre do 5º ano
do Mestrado Integrado em Engenharia Informática e Computação (MIEIC) da Faculdade de
Engenharia da Universidade do Porto (FEUP).
A aplicação tem como principal objetivo obter grandes dimensões de informação musical,
interpretando-a ao nível do input do utilizador e representá-la de acordo com a sua
componente geográfica e métrica, oferecendo ao utilizador possíveis estudos e curiosidades
sobre os seus interesses musicais.

\end{abstract}

%----------------------------------------------------------------------------------------
%	ARTICLE CONTENTS
%----------------------------------------------------------------------------------------

\section{Introdução}\label{sec:intro}

%------------------------------------------------

A música está presente no quotidiano de milhões de pessoas, seja através de dispositivos
móveis, como smartphones ou leitores de música portáteis, ou através do computador,
sistema de hi-fi ou até televisão. É indiscutível a importância e valor que esta arte tem.
Contudo, numa sociedade sedenta de informação, e num ecossistema como a Internet, onde a
procura de informação é desde hobbie até ao modelo de negócio de várias empresas, é
bastante comum, até para o casual ouvinte de música, procurar mais informação sobre o
artista que está a ouvir, como por exemplo: letra da música, outros trabalhos do artista,
videoclip da música, biografia da banda, etc.
O desenvolvimento de uma aplicação, onde a música será o domínio no qual nos vamos focar,
usando datasets disponíveis no \textit{Spotify} e no \textit{SongMeanings}, através da
plataforma \textit{The Echo Nest}, que permitirá aos utilizadores pesquisas mais alargadas, de
forma a satisfazer as suas necessidades e curiosidades, aumentando a experiência para lá
do sentido auditivo.
Ao longo deste documento podem-se encontrar as seguintes secções:
\begin{compactitem}
  \item Estado da arte: pequena descrição daquilo que já existe no mercado, semelhante
    ao que vai ser desenvolvido, e os aspectos diferenciadores desta aplicação;
  \item Fontes de informação: apreciação da autoridade da fonte e da qualidade dos dados;
  \item Estrutura da informação e datasets;
  \item Modelo conceptual do domínio;
  \item Tarefas de pesquisa: identificação de algumas tarefas de pesquisa a fazer sobre
    os dados (pesquisas tipo, cenários de utilização);
  \item Conclusões;
  \item Referências;
\end{compactitem}

%------------------------------------------------

\section{Estado da Arte}\label{sec:art}

Actualmente, o Spotify possui os seguintes campos de pesquisa:
\begin{compactitem}
  \item pesquisa por artista, faixa, álbum ou ano;
  \item pesquisa refinada por AND, OR e NOT, como por exemplo:
    \begin{compactitem}
      \item Zeppelin OR Floyd: lista todos os resultados com as palavras-chave
        “Zeppelin” ou “Floyd”;
      \item Metallica NOT Anger: lista todas as faixas dos Metallica, excepto as que têm
        a palavra “Anger”;
    \end{compactitem}
  \item pesquisa por género musical;
  \item pesquisa por label;
  \item pesquisa por isrc: apresenta todas as faixas correspondentes ao ID, de acordo com
    o International Standard Recording Code;
  \item pesquisa por upc: apresenta todos os álbuns correspondentes ao ID, de acordo com
    o Universal Product Code;
  \item pesquisa por tag:new: lista os álbuns adicionados mais recentemente;
  \item Referências;
\end{compactitem}


A aplicação \textit{Spotify on the Rocks} diferencia-se relativamente à aqui descrita,
na medida em que permite a combinação de vários campos de pesquisa, para além das
acima mencionadas.
Além disso, a aplicação integra a API do \textit{Spotify} com o \textit{SongMeanings}
(recorrendo à plataforma \textit{The Echo Nest}) e com a API do Youtube, para que o
utilizador possa ter lyrics e videoclipes associados às músicas.


%------------------------------------------------

\section{Fontes de informação}\label{sec:sources}

Spotify é, atualmente, a maior plataforma musical online e é famosa pela quantidade de
informação que possui sobre o negócio, mantendo além da informação sobre artistas, álbuns
e respetivas faixas, imensa informação bastante precisa sobre os géneros musicais, origem
geográfica e cronológica das obras e, limitado a quem possui uma conta (gratuita), informação
sobre o histórico de utilizadores, bem como as preferências e construções dos demais que
constituem a comunidade virtual.

A empresa, que é sediada em Estocolmo, assinou acordos com as gravadoras Universal Music,
Sony BMG, EMI, Hollywood Records e Warner Music, entre outros. O serviço tinha em 15 de
setembro de 2010 quase 10 milhões de utilizadores. Em março de 2012, tinha cerca de 3 milhões
de utilizadores pagos. Ainda em 2012, o serviço foi premiado na décima sexta edição do Webby
Awards, como site mais importante.

O The Echo Nest é uma plataforma que agrega diferentes bases de dados de cerca de 30 milhões
de músicas, que utiliza técni

SongMeanings não é um site de lyrics como os outros: é uma comunidade de milhares de amantes
de música, que, além de contribuiremm com lyrics, discutem e comentam sobre os significados
e mensagens subjacentes de determinadas canções.
Em setembro de 2011, a SongMeanings concordou com os termos da LyricFind, licenciando mais
de um milhão de lyrics. Este acordo faz da SongMeanings uma entidade legal, entre as centenas
de sites de letras de músicas ilegais, para além de permitir ter letras exactas.


%------------------------------------------------

\section{Estrutura de informação e \textit{datasets}}\label{sec:structure}

Toda a informação do Spotify está disponível sob forma de uma Application Programming Interface
(API) online, seguindo arquitetura REST (Representational State Transfer), oferecendo uma
colossal fonte de informação sobre artistas, álbuns, faixas e gêneros músicais, bem como a
recursos cronológicos sobre os trabalhos, informações geográficas dos artistas, estilos
associados e bandas relacionadas, bem como preferências e listas de reprodução construídas
pelos utilizadores.
A informação proveniente das pesquisas serão guardadas pela aplicação de forma a facilitar
a combinação com outros serviços, produzindo análises com teor analítico, facilitando o
estudo ou a descoberta de curiosidades fruto do cruzamento de dados.
A aplicação é responsável por recolher uma amostra de dados baseadas na informação introduzida
pelo utilizador ou, por omissão, descrever as últimas pesquisas efetuadas.
Embora a riqueza do serviço, apenas parte da API é usada para a aplicação.

%------------------------------------------------

\section{Modelo conceptual de domínio}\label{sec:concept_model}

%------------------------------------------------

\section{Tarefas de pesquisa}\label{sec:searches}

Com a aplicação anteriormente descrita, pretendemos que seja possível efectuar vários
tipos de pesquisa/operações sobre os dados que vamos utilizar, tais como:
\begin{compactitem}
  \item Procurar músicas por título, artista, álbum, género, país, ano ou uma combinação
    de qualquer destes campos, por exemplos:
    \begin{compactitem}
      \item procurar uma música do género Blues, cujo artista seja da Inglaterra;
      \item procurar uma música que seja um trabalho conjunto de 2 artistas;
    \end{compactitem}

  \item Obter uma lista de músicas através de combinações de pesquisas, podendo com
    isto criar uma playlist, se o utilizador estiver autenticado, como por exemplo:
    \begin{compactitem}
      \item uma playlist com 10 músicas dos Metallica e dos Muse;
      \item uma playlist com músicas portuguesas, ou com músicas portuguesas e brasileiras
    \end{compactitem}

  \item Utilizar a informação de uma música para obter a letra (lyrics) a esta associada,
    através do SongMeanings. Para além disso, podemos apresentar as discussões feitas
    acerca do significado dessa mesma música, uma vez que é esse o principal objectivo
    do \textit{SongMeanings};
  \item Tirar partido das playlists dos utilizadores, para ver quantos followers o
    spotify tem, e representá-los geograficamente (por exemplo, através de um mapa ou
    de um gráfico)
  \item Obter, através de toda esta informação, uma representação gráfica de dados
    estatísticos, como por exemplo:
    \begin{compactitem}
      \item número de utilizadores portugueses registados no spotify
      \item músicas mais ouvidas ou que são mais vezes avançadas (\textit{skipped})
      \item género musical mais apreciado pelos utilizadores
      \item \textit{word counter} em pesquisas conjuntas, seja ao nível de faixa, album,
        descrição, etc
    \end{compactitem}
\end{compactitem}

Consideramos que estes cenários de utilização vão ser bastante úteis, uma vez que o
spotify não possibilita pesquisas num âmbito tão alargado.

%------------------------------------------------

\section{Conclusões}\label{sec:conclusions}

%----------------------------------------------------------------------------------------
%	REFERENCE LIST
%----------------------------------------------------------------------------------------

%% auto bibliographic list
%\renewcommand{\bibname}{Referências}
% uses bibtex file
\bibliography{refs}
% format for PT/EN alpha/unsrt
%\bibliographystyle{alpha-pt}
%\bibliographystyle{alpha}
%\bibliographystyle{unsrt-pt}
%\bibliographystyle{unsrt}

%----------------------------------------------------------------------------------------

\end{document}
